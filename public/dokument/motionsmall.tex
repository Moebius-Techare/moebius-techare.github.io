\documentclass[12pt,a4paper]{article}
\usepackage[utf8]{inputenc}
\usepackage[swedish]{babel}
\usepackage{amsmath}
\usepackage{amsfonts}
\usepackage{amssymb}
\usepackage{graphicx}
\usepackage{mathrsfs}
%\usepackage[left=2cm,right=2cm,top=2cm,bottom=2cm]{geometry}
\author{*Vem är du som skrivit motionen?*}
\begin{document}

\title{Motion angående \\ grejer}
\date{*Datum motionen är skriven*}
% Om raden \date tas bort så blir det automatiskt dagens datum
\maketitle

\section*{Bakgrund}
Här skriver man bakgrunden till motionen, varför har du skrivit den och varför bör mötet godkänna den?


\section*{Yrkande}

Mot denna bakgrund yrkar jag\ldots\\

\textit{\textbf{att}} *vad du vill ska hända*\\

eller om man vill yrka flera saker:\\

\begin{enumerate}
  \item \textit{\textbf{att}} en grej
  \item \textit{\textbf{att}} annan grej
\end{enumerate}

% Bra formuleringar är: "Möbius ska köpa in xxx" eller "ändra bla bla bla"
% eller uppdra styrelsen att bla bla bla

\end{document}
